\documentclass[12pt, a4paper]{report}
\usepackage[left=2cm, right=1.5cm, top=2cm, bottom=2cm, bindingoffset=0cm]{geometry} % поля
\usepackage{indentfirst} % отступ после заголовка
% \setlength{\parindent}{5ex} % отступ в начале абзаца
% \setlength{\parskip}{1em} % отступ между абзацами

\usepackage[utf8]{inputenc} % кодировка
\usepackage[english,russian]{babel} % язык
\usepackage{cmap} % поиск в pdf

\usepackage{tikz} % выделение текста (` `)
\tikzset{baseline, inner sep=2pt, minimum height=12pt, rounded corners=2pt}
\newcommand{\code}[1]{\mbox{\ttfamily \tikz \node[anchor=base,fill=black!12]{#1};}}

\makeatletter % убираю "Глава №"
\def\@makechapterhead#1{
    {\normalfont
\bfseries\begin{center}\lowercase{#1}\end{center}\par\nobreak\vskip 10\p@}}
\makeatother

\usepackage{graphicx} % картинки
\graphicspath{ {images/} } % место по умолчанию

\usepackage{amsmath} % математика

\title{Тетрадь}
\author{Цыпышев Тимофей, ИУ5-41Б \thanks{Семинары ведёт едведева К.О.}}
\date{8 февраля 2024 г. - \today}

\begin{document}
\maketitle
\tableofcontents

\part{Module 1}

\chapter{Seminar 1}

\section{Exercise №1}

\subsection{Match the words (1-6) with their definitions (a-f). Use a dictionary if necessary.}
\begin{enumerate}
    \item[1.] stimulated
    \item[2.] radiation
    \item[3.] acronym
    \item[4.] emission
    \item[5.] beam
    \item[6.] amplification
\end{enumerate}

\begin{enumerate}
    \item[a.] energy in the form of heat or light that you cannot see and
        which can be very harmful
    \item[b.] a word formed from the initial letters of other words
    \item[c.] the increase in volume of a signal
    \item[d.] a line of radiation or particles flowing in one direction
    \item[e.] the act of sending out gases or other substances
    \item[f.] made stronger or more active
\end{enumerate}

\subsection{Solution}

\begin{enumerate}
    \item[1.] f
    \item[2.] a
    \item[3.] b
    \item[4.] e
    \item[5.] d
    \item[6.] c
\end{enumerate}

\section{Exercise №2}

\subsection{In groups answer the questions.}
\begin{enumerate}
    \item[1.] What is a laser?
        \begin{enumerate}
            \item[a.] a device which produces a very narrow beam of light useful in many technologies
            \item[b.] a process of optical amplification of light based on radiation emission
            \item[c.] both a and b
        \end{enumerate}
    \item[2.] What kind of word is the word ‘laser’?
        \begin{enumerate}
            \item[a.] acronym
            \item[b.] shortening
            \item[c.] contraction
        \end{enumerate}
    \item[3.] Can you decode the word ‘laser’? (use the words from task 1)
        \begin{enumerate}
            \item[] L... A... by Stimulated E... of R... .
        \end{enumerate}
\end{enumerate}

\subsection{Solution}
\begin{enumerate}
    \item[1.] a
    \item[2.] a
    \item[3.] Light Amplification by Stimulated Emission of Radiation
\end{enumerate}

\section{Exercise №3}

\subsection{Study the pictures below. Which of the following words and phrases refer to ordinary light
    (1) and which to laser light (2)?}

Coherent; its intensity decreases with distance; highly monochromatic; it is not strictly
monochromatic; organised; less intense; travels in one direction; incoherent; highly intense;
concentrated; travels in all directions; disorganised.

\subsection{Solution}
\code{Ordinary light:}
\begin{itemize}
    \item disorganized
    \item its intensity decreases with distance
    \item it is not strictly monochromatic
    \item less intense, incoherent
    \item travels in all directions
\end{itemize}

\code{Laser light:}
\begin{itemize}
    \item organized
    \item Coherent
    \item highly monochromatic
    \item travels in one direction
    \item highly intense
    \item concentrated
\end{itemize}

\section{Exercise №6}
\code{[Устно]}
\subsection{Read the text again and answer the following questions.}
\begin{enumerate}
    \item Why can we say that lasers were predicted long before their invention?
    \item What is a laser? What does the word ‘laser’ mean?
    \item What kind of beam do lasers have?
    \item What do we mean by the words ‘monochromatic, directional, and coherent’ when we refer
          to laser light?
    \item Why is the light from the laser so concentrated?
    \item Who proposed the theoretical possibility of the process that made lasers possible?
    \item Who created the first microwave generator?
    \item Who demonstrated the first successful light laser?
    \item What laser types are mentioned in the text?
    \item Do you agree with the author’s opinion that lasers have found myriads of useful
          applications? What examples do you think best prove this point?
    \item While reading this text, which uses of lasers surprised you the most?
    \item Can you think of an example of a laser device or technology that you have used or are using?
\end{enumerate}

\section{Exercise №7}
\code{[Устно]}
\subsection{Read the statements and decide which of them are true (T) and which are false (F)
    according to text 10A. Explain why.}
\begin{enumerate}
    \item The word ‘laser’ means microwave amplification by stimulated emission of radiation.
    \item Laser was invented at the dawn of the 20th century.
    \item Albert Einstein was the first inventor of a laser.
    \item Laser came into existence only in the second half of the 20-th century.
    \item Unfortunately most of the applications of a laser proved to be unattainable in the real world.
    \item The use of lasers in thermonuclear fusion reactors may be the key to the future.
    \item Laser weapons are widely used by the military.
    \item In medicine lasers can be used for various surgical procedures.
    \item Very few inventions can match the impact of the laser’s invention.
    \item Laser technology has a promising future.
\end{enumerate}

\end{document}