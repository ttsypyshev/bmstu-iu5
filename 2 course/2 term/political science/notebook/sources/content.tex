\part{Модуль №1}

\chapter{Семинар №1 09.02.24}

\section{Введение}

\textbf{Политология} - наука о политике, полит. власти, формировании и развитии полит. отношений.

Политика, подобно кровеносным капиллярам, пронизывает все сферы общества - вне политике не может быть никакой деятельности.
Если человек не занимается политикой, то Политика всё равно занимается им.

\subsection*{Творческая часть}
Выходим с планом, рассказываем новости недели.

\subsubsection{Источники новостей:}
\begin{itemize}
    \item Газеты (\allocation{Аргументы и факты}, \allocation{Аргументы и недели})
    \item Телевизор (\allocation{Россия 1}, \allocation{1 программа})
\end{itemize}

КР - 29.03.24
Реферат - 24.05.24 (можно раньше)

\section{Идейные истоки и сущность политической науки}

\subsubsection{Вопросы:}
\begin{enumerate}
    \item Взгляды мыслителей о политическом устройстве общества в древней Индии и древнем Китае(Конфуцианства)
    \item Политическая мысль античного общества (Платон. Аристотель, Цицирон)
    \item Особенности религиозного влияния на политику средневековой Европы
    \item Развитие политических воззрений в эпоху возрождения (Никола Макиавелли)
    \item Политические взгляды мыслителей нового времени и французских просветители (Эпоха просветления)
    \item Эволюция политических теорий в России
    \item Сущность политической науки (политология):
          \begin{itemize}
              \item Связь политологии с другими науками
              \item Функции политологии
          \end{itemize}
    \item Функции политологии
\end{enumerate}