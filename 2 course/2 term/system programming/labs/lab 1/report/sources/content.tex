\section{Цель выполнения лабораторной работы}
Целью выполнения лабораторной работы №1 является знакомство с специальными
электронными справочниками системного программиста и изучение принципов поиска
них информации по операционным системам, предназначенной для системного программиста.

\section{Порядок и условия проведения работы}
\begin{enumerate}
    \item Изучить в общем пособии (разделы 1,5,7) [6 – см. на сайте] разделы по: работе в режиме
          командной строки (КС) и по работе с файл-менеджерами (ФМ).
    \item В режиме КС запустить команды: DIR, HELP, DATE и SET. Продемонстрировать полученные
          навыки преподавателю ЛР.
    \item В программах FAR или VC (архив TASM3.ZIP – где 3 ЛР – на сайте) проверить: переключение
          каталогов, поиск файлов, создание и редактирование простого текстового файла, копирование
          и перемещение файлов, навигацию по меню. Продемонстрировать полученные навыки преподавателю ЛР.
    \item Скачать и развернуть справочники под эмулятором ОС (DOSBox v 7.4 – если на своем компьютере
          он не установлен, то скачать с сайта, установить, русифицировать и смонтировать виртуальный диск V:
          - см. ниже ) или в КС под CMD.EXE.
    \item Ответить устно на все контрольные вопросы ЛР.
    \item Изучить таблицу заданий для своего варианта
    \item Найти свою информацию по своему варианту и зафиксировать в отчете и изучить.
          СП 2024 год 2 курс ИУ5- 4-й сем. и 3-й курс ГУИМЦ 6-й семестр Большаков С.А.
    \item Изучить контрольные вопросы к ЛР и ответить на них.
    \item Показать ее преподавателю найденную информацию (демонстрация - отмечается в журнале)
    \item Оформить и распечатать отчет по своему варианту (шаблон в архиве этой ЛР).
    \item Защитить ЛР у преподавателя по контрольным вопросам (защита - отмечается в журнале
          и на титульном листе отчета).
\end{enumerate}

\section{Краткая инструкция по работе со справочником Help6}
 (Вставляются результаты поиска в виде копии текста из окна командной строки или снятых с
 экрана скриншотов. Даются пояснения и примеры.)

\section{Результаты поиска команды ОС}


\section{Результаты поиска прерывания ОС}
(Краткое назначение прерывания ОС Вставляются результаты поиска в виде копии текста из
окна командной строки или снятых с экрана скриншотов(без черного!!!). Даются пояснения и
примеры.)

\section{Результаты поиска управляющего блока ОС}
(Краткое назначение управляющего блока ОС Вставляются результаты поиска в виде копии
текста из окна командной строки или снятых с экрана скриншотов(без черного!!!). Даются
пояснения и примеры.)

\section{Выводы по ЛР}
В результате выполнения лабораторной работы была освоена работа с тремя справочниками,
получены навыки оперативного поиска информации о нужных командах, прерываниях и блоках
управления.

yftdrtdrt