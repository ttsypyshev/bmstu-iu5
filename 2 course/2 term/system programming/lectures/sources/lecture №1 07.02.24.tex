\documentclass[12pt, a4paper]{report}
\usepackage[left=2cm, right=1.5cm, top=2cm, bottom=2cm, bindingoffset=0cm]{geometry} % поля
\usepackage{indentfirst} % отступ после заголовка
% \setlength{\parindent}{5ex} % отступ в начале абзаца
% \setlength{\parskip}{1em} % отступ между абзацами

\usepackage[utf8]{inputenc} % кодировка
\usepackage[english,russian]{babel} % язык
\usepackage{cmap} % поиск в pdf

\usepackage{graphicx} % картинки
\graphicspath{ {images/lecture 1/} } % место по умолчанию

\usepackage{amsmath} % математика

\usepackage{tikz} % выделение текста (` `)
\tikzset{baseline, inner sep=2pt, minimum height=12pt, rounded corners=2pt}
\newcommand{\code}[1]{\mbox{\ttfamily \tikz \node[anchor=base,fill=black!12]{#1};}}

\title{Лекция №1}
\author{Цыпышев Тимофей, ИУ5-41Б \thanks{Лекции вёл Большаков С.А.}}
\date{07.02.24}

\begin{document}
% \maketitle
% \tableofcontents

\chapter{Системное программирование}

\section{Введение}

\subsection{Данные преподавателя}

Методический сайт: \code{www.sergebolshakov.ru}

Почта: \code{sergebolremote@gmail.com}

\subsection{Данные обучения}

Лекции СП 2 часа в неделю. Зачёт по ЛР и лекциям

ЛР СП 2 часа в неделю. Зачёт по курсу

КР СП 51 час в семестр. (ТЗ - 4я нед., КР - 14 нед.). Диф. зачёт

РК. На 8 и 16 неделе. Зачёт

\subsection{ЛР курса СП}

\textbf{Лабораторные работы за 4 семестр:}
\begin{enumerate}
    \item \code{Справочник ОС} (МУ, DOC, Задание, Теория, Контроль) - 2 ч.
    \item \code{Программирование КФ} (МУ, DOC, Задание, Теория, Контроль), КС 4 ч.
    \item \code{Ассемблер} (МУ, DOC, Задание, Теория, Контроль)
    \item \code{Циклы переводов в hex} (МУ, DOC, Задание, Теория, Контроль)
    \item \code{Буферизация и перевод в} (МУ, DOC, Задание, Теория, Контроль)
    \item \code{Параметры командной строки} (МУ, DOC, Задание, Теория, Контроль)
    \item \code{Ввод адреса} (МУ, DOC, Задание, Теория, Контроль)
    \item \code{Вывод дампа} (МУ, DOC, Задание, Теория, Контроль)
    \item \code{Макроассемблер} (МУ, Задание, Теория, Контроль)
\end{enumerate}

\subsection{КР СП}

\textbf{Требования и правила выполнения КР:}
\begin{itemize}
    \item \code{КР содерижит:} 3 листа формата A1 (при специальных ограничениях можно распечатать на А4) примеры на сайте. Результат - дифф. зачет по КР СП.
    \item \code{КР предоставляется в сеставе:} 3 листов, ПС(2-х мод) и Документов (7-мя).
    \item \code{Документы КР (как на 1-м курсо):} ТЗ, ТО, ОП, РП, РСП, ПМИ, ПО в исходном и исполнимом форматах (на сайте есть примеры/образцы листов и документов) (CM)
    \item \code{Сдача ПО КР} с ПМИ, только на основе испытаний правильной документации и работающего ПО (TSR), (можно на своем комп.)
    \item \code{Сроки сдачи КР} не позднее 14 недели семестра. Защита КР СП.
    \item \code{Оценка по КP} по 5-бальной системе (она идет в диплом бакалавра)
    \item \code{Выполниется индивидуально} по своему варианту с учетом индекса группы.
    \item \code{Т3 по КР предоставляются и подписывается до 4-й неделе} (На Конс. КР СП или ЛР), ТЗ - это домашнее задание (ДЗ КР СП), Т3 и КР СП принимает только Большаков С.А.
    \item Если ТЗ вовремя не подписано и правильно не оформлено, то минус 1 балл!!!
    \item На сайте есть обобщенный (большой) вариант КР, и два варианта упрощенных (порты и буфер клавиатуры),
    \item По КР СП. будут отельные лекции (СМ), Общие вопросы ОП (СМ)
    \item В обобщённых МУ по дисциплине СП (сл, сайт), разобраны все темы КР - Глава 10.
    \item Литература по КP СП (CМ) Содержания HP (СМ)
\end{itemize}

\subsection{Сайт}

Методический ресурс дисциплины: \code{http://sergebolshakov.ru/} 

Пользователь: \code{ИУ5-41} 

Пароль: \code{14-5УИ} 

\end{document}